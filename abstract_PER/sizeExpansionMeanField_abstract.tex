% \documentclass[sigconf]{acmart}
\documentclass{sig-alternate-per-Performance2018}


\usepackage{calc,xcolor}
\usepackage[utf8]{inputenc}
\usepackage[T1]{fontenc}

\usepackage{amsmath,amsfonts,theorem}
\newtheorem{theorem}{Theorem}
\newtheorem{lemma}{Lemma}
\newtheorem{coro}[theorem]{Corollary}
\usepackage{tikz}
%\usepackage[backend=bibtex,firstinits=true,maxbibnames=99]{biblatex}
%\bibliography{biblio}
\usepackage{hyperref}
% \definecolor{darkblue}{rgb}{0 0 .6}
% \hypersetup{colorlinks=true,linkcolor=darkblue,citecolor=darkblue,urlcolor=darkblue}
% \hypersetup{pageanchor=false}
%\hypersetup{draft=true}
%\usepackage{listings}

\newcommand\XN{X^{(N)}}
\newcommand\LN{L^{(N)}}
\newcommand\DeltaN{\Delta^{(N)}}
\newcommand\bl{{\text{\boldmath$\ell$}}}
\newcommand\betaN{\beta^{(N)}}
\newcommand\PsiN{\Psi^{(N)}}
\newcommand\E{\mathcal{E}}
\newcommand\N{\mathbb{N}}
\newcommand\R{\mathbb{R}}
\newcommand\Z{\mathbb{Z}}
\newcommand\calL{\mathcal{L}}
\newcommand\floor[1]{\left\lfloor#1\right\rfloor}
\newcommand\var[1]{\mathrm{var}\left[#1\right]}
\newcommand\svar[1]{\mathrm{var}[#1]}
\newcommand\esp[1]{{\mathchoice{\besp{#1}}{\sesp{#1}}{\sesp{#1}}{\sesp{#1}}}}
\newcommand\besp[1]{\mathbb{E}\left[#1\right]}
\newcommand\sesp[1]{\mathbb{E}[#1]}
\newcommand\espN[1]{{\mathchoice{\bespN{#1}}{\sespN{#1}}{\sespN{#1}}{\sespN{#1}}}}
\newcommand\bespN[1]{\mathbf{E}^{(N)}\left[#1\right]}
\newcommand\sespN[1]{\mathbf{E}^{(N)}[#1]}
\newcommand\Proba[1]{{\mathchoice{\bProba{#1}}{\sProba{#1}}{\sProba{#1}}{\sProba{#1}}}} 
\newcommand\bProba[1]{\mathbf{P}\left[#1\right]}
\newcommand\sProba[1]{\mathbf{P}[#1]}
\newcommand\norm[1]{{\mathchoice{\bnorm{#1}}{\snorm{#1}}{\snorm{#1}}{\snorm{#1}}}}
\newcommand\bnorm[1]{\left\|#1\right\|}
\newcommand\snorm[1]{\|#1\|}

\newcommand\abs[1]{\left|#1\right|}
\newcommand\p[1]{\left(#1\right)}
\newcommand\dt{\frac{d}{dt}}
\newcommand\ds{\frac{d}{ds}}
\newcommand\Sym{\mathrm{Sym}}
\newcommand\red[1]{{\color{red!70!black!100}#1}}

\newcommand\J[1]{J_{(#1)}}


\graphicspath{{}{../simulations/output_pdfs/}}
%\pdfsuppresswarningpagegroup=1

\begin{document}
\conferenceinfo{IFIP WG 7.3 Performance 2018}{~~~Toulouse, France}

\title{Size Expansions of Mean Field Approximation: Transient and
  Steady-State Analysis}%
\subtitle{Extended abstract\thanks{The full version is available at
    \protect\cite{githubPaper2018}.}}

\numberofauthors{3}
\author{
  \alignauthor
  Nicolas Gast\\
  \affaddr{Inria, France}\\%
  \affaddr{Univ. Grenoble Alpes, LIG}\\%
  \affaddr{Grenoble, France}
  \email{nicolas.gast@inria.fr}
  \alignauthor
  Luca Bortolussi\\
  \affaddr{DMG, University of Trieste}\\
  \affaddr{Italy}\\
  \email{lbortolussi@units.it}
  \alignauthor
  Mirco Tribastone\\
  \affaddr{IMT School for Advanced Studies Lucca}\\
  \affaddr{Italy}\\
  \email{mirco.tribastone@imtlucca.it}
}

\date{\today}

\maketitle 


\begin{abstract}
  Mean field approximation is a powerful tool to study the performance
  of large stochastic systems that is known to be exact as the
  system's size $N$ goes to infinity. Recently, it has been shown
  that, when one wants to compute expected performance metric in
  steady-state, mean field approximation can be made more accurate by
  adding a term in $1/N$ to the original approximation. This is called
  the \emph{refined} mean field approximation in
  \cite{gast2017refined}.  In this paper, we show how to obtain the
  same result for the transient regime and we provide a further
  refinement by expanding the term in $1/N^2$ (both for transient and
  steady-state regime). Our derivations are inspired by moment-closure
  approximation. We provide a number of examples that show this new
  approximation is usable in practice for systems with up to a few
  tens of dimensions.
\end{abstract}

% % A category with the (minimum) three required fields
% \category{G.3}{Mathematics of Computing}{Probability and Statistics} [Queueing Theory]
% \terms{Theory, Performance}
% \keywords{Algorithms, complexity, open problems, approximation}


\section{Introduction}

Mean field approximation is a widely used technique in the performance
evaluation community (\emph{e.g.}, to study load-balancing strategies
\cite{mitzenmacher1996power,vvedenskaya1996queueing} to mention a
popular application). The focus of this approximation is to study the
performance of systems composed of a large number of interacting
objects.  This approximation can be used to study transient (for
example the time to fill a cache \cite{gast2015transient_short}) or
steady-state properties (for example the steady-state response time of
a system \cite{mitzenmacher1996power,vvedenskaya1996queueing}).  One
of the reasons of the success of mean field approximation is that it
is often very accurate as soon as $N$, the number of objects in the
system, exceeds a few hundreds. In fact, this approximation can be
proven to be asymptotically exact as $N$ goes to infinity, see for
example \cite{kurtz70,benaim2008class} and explicit bounds for the
convergence rate exist
\cite{bortolussi2013bounds,gast2017expected,ying2016rate}.

Recently, the authors of \cite{gast2017refined} proposed what they
call a \emph{refined} mean field approximation that can be used to
characterize steady-state performance metrics more precisely. Their
refinement uses that for many models, a steady-state expected
performance metric of a system with $N$ objects $\esp{h(X)}$ is equal
to its mean field approximation $h(\pi)$ plus a term in $1/N$:
\begin{align}
  \label{eq:refined_1/N}
  \esp{h(X)} = h(\pi) + \frac1NV_{(h)} + o\p{\frac1N},
\end{align}
where $\pi$ is the fixed point of the ODE that describes the mean
field approximation and $V_{(h)}$ is a constant that can be easily
evaluated numerically.  By using a number of examples, they show that
the refined approximation $h(\pi) + \frac1NV_{(h)}$ is much more
accurate than the mean field approximation for moderate system sizes
(i.e., a few tens of objects).

\section{Main results}

In \cite{githubPaper2018}, we extend this method in two directions:
First we generalize Equation~\eqref{eq:refined_1/N} to the transient
behavior; second we establish the existence of a second order term in
$1/N^2$ (both in transient and steady-state regimes). More precisely,
we establish conditions such that for any smooth function $h$, there
exist constants $V_{(h)}$ and $A_{(h)}$ such that for any time
$t\in[0;\infty)\cup\{\infty\}$:
\begin{align*}
  %\label{eq:refined_1/N^2}
  \esp{h(X(t))} = h(x(t)) + \frac1N V_{(h)}(t) + \frac1{N^2} A_{(h)}(t) +
  o\p{\frac1{N^2}}.  
\end{align*}
We show that for the transient regime $t\in[0,\infty)$, $V_{(h)}(t)$
and $A_{(h)}(t)$ satisfy a linear time-inhomogeneous differential
equation that can be easily integrated numerically. For the
steady-state, the constants $V_{(h)}(\infty)$ and $A_{(h)}(\infty)$
correspond to the fixed point of this differential equation.

We use the above expansion to propose two new approximations that
depend on the system size $N$ and that are expansions of the classical
mean field approximation to the order $1/N$ and $1/N^2$. This gives
three approximations:
\begin{itemize}
\item The mean field approximation: $h(x(t))$;
\item $1/N$-expansion: $h(x(t))+V_{(h)}(t)/N$;
\item $1/N^2$-expansion: $h(x(t))+V_{(h)}(t)/N+A_{(h)}(t)/N^2$.
\end{itemize}

We derive our results for the classical model of density-dependent
population process \cite{kurtz70}.

\section{Numerical comparison}

In \cite{githubPaper2018}, we study numerically the accuracy of the
approximation by considering three examples: two malware propagation
models and the classical supermarket models of
\cite{mitzenmacher1996power,vvedenskaya1996queueing}. Our numerical
results shows that the two expansions very accurately capture the
transient behavior of such a system even when $N\approx10$. Moreover,
they are generally much more accurate than the classical mean field
approximation for small values of $N$ (for transient and steady-state
regimes).  Our experiments also confirm the good accuracy of the
$1/N$-expansion approximation that was observed for steady-state
values in \cite{gast2017refined}: In most cases, the largest gain in
accuracy comes from the $1/N$-term (both for the transient and
steady-state values).  The $1/N^2$-term does improve the accuracy but
only marginally.  We also study the limit of the method by studying an
unstable mean field model that has an unstable fixed point.

\paragraph*{The Supermarket Model}
To give a flavor of the results, let us consider the supermarket model
(more examples can be found in the full paper).  The system is
composed of $N$ identical servers. Jobs arrive at a central broker
according to a Poisson process of rate $\rho N$ and are dispatched
towards the servers by using the JSQ(2) policy: for each incoming job,
the broker samples $2$ servers at random and sends the jobs to the
server that has the smallest number of jobs in its queue (ties are
broken at random). The time to process a job is exponentially
distributed with mean $1$.

\begin{figure}[ht]
  \centering
  \begin{tabular}{c}
    \includegraphics[width=.8\linewidth]{twoChoice_refRefTransient_N10_rho90}
  \end{tabular}
  \caption{Supermarket model : expected queue length as a function of
    time. }
  \label{fig:supermarket-transient}
\end{figure}

In Figure~\ref{fig:supermarket-transient}, we plot how the expected
queue length evolves with time for the supermarket model with $N=10$
servers and $\rho=0.9$.  We start in a system where $8$ queues have
$3$ jobs and $2$ queues have two jobs.  We observe that the two
expansions are much more accurate than the classical mean field
approximation and that the $1/N^2$-expansion provides a slightly
better approximation than the $1/N$-expansion.


In Table~\ref{tab:supermarket}, we show the average queue length for
different values of $\rho$ and $N$. Again, we observe that the values
coming from the two expansions are much closer to the values obtained
by simulation than the classical mean field approximation. Again, most
of the gain seems to come from the $1/N$-term. 

\begin{table}[ht]
  \centering
  \small 
  \begin{tabular}{@{}|@{ }ccc@{ }|@{}c@{}|@{}c@{}|@{}c@{}|@{}c@{}|@{}}
    \hline
    $N$&$k$&$\rho$&Mean field&$1/N$-expansion&$1/N^2$-expansion&Simulation\\
    \hline 
    10 & 2 & 0.9 	 &2.3527 &	2.7513 &	2.8045 &	2.8002 \\
    20 & 2 & 0.9 	 &2.3527 &	2.5520 &	2.5653 &	2.5662 \\
    \hline
    10 & 2 & 0.95 	 &3.2139 &	4.1017 &	4.3265 &	4.2993 \\
    20 & 2 & 0.95 	 &3.2139 &	3.6578 &	3.7140 &	3.7124 \\
    \hline  \end{tabular}
  \caption{Supermarket model, steady-state average queue length :
    comparison of the value computed by simulation with the three
    approximations. }
  \label{tab:supermarket}
\end{table}



\section{Conclusion}

To summarize, we show in \cite{githubPaper2018} how mean field
approximation can be refined by a first term in $1/N$ and a second
term $1/N^2$ where $N$ is the size of the system. We exhibit
conditions that ensure that this asymptotic expansion can be applied
for the transient as well as the steady-state regimes. In the
transient regime, these constants satisfy ordinary differential
equations that can be easily integrated numerically.  We provide a few
examples that show that the $1/N$ and $1/N^2$ expansions are much more
accurate than the classical mean field approximation.  We also study
the limitations of the approach and show that, when the mean field
approximation does not have an attractor, these new approximations
might be unstable for large time horizons.  Obtaining a better
approximation in this case remains a challenge that we leave for
future work.

Finally, we show that, despite the complexity of the formulas, it is
relatively easy to compute the $1/N$ and $1/N^2$ terms (in the
transient and steady-state regimes) for realistic models such as the
supermarket model. The developed method is generic and is implemented
in a tool \cite{rmfTool2018}.


\paragraph*{Reproducibility} The code to reproduce the paper --
including simulations, figures and text -- is available at
\cite{githubPaper2018}.


\small 
\bibliographystyle{plain}
\bibliography{../paper/biblio}



\end{document}


%%% Local Variables:
%%% TeX-source-correlate-method-active: synctex
